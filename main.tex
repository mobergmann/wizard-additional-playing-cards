\documentclass[3pt]{article}

% One Wizard Playing Card has the dimensions (width: 55mm, height: 87), the manual should be double that.
\usepackage[paperwidth=110mm, paperheight=174mm]{geometry}
\usepackage{parskip}

\begin{document}

\section{Extra Wizard Rules}
Wizard ist ein kompetetives Kartenspiel, in dem die Lehrlinge (du und deine Mitspieler) versuchen die Zukunft vorauszusagen, indem sie schätzen wie viele Stiche sie bekommen.
Wenn deine Vorhersage Stimmt, bekommst du Punkte und diejenige Person mit den meisten Punkten am Ende hat gewonnen.

\section{Grundregeln}
...

\subsection{Karten}
...

\subsection{Begriffe}
...

\subsection{Spieleweise}
...

\section{Sondertegeln}
Um das Spiel interesannter zu gestalten und den Schwierigkeitsgerad anzuheben gibt es in der offiziellen Wizard Sonderedition einige weitere Karten.
Diese machen das vorhersagen der Stiche schwiereiger und sorgt so für mehr Vielfallt und somit auch mehr Spaß beim spielen.

\subsection{Drache}
Der Drache ist die stärkste Karte im Spiel, auch stärker als ein Zauberer.
Lediglich wenn der Drache mit einer Fee in einem Stich liegt ist der Drache nicht mehr die Stärkste Karte.
Führ mehr Informationen, siehe das Kapitel \ref{cha:fairy}.

\subsection{Fee}\label{cha:fairy}
Die Fee ist die schwächste Karte in dem Spiel, sogar schwächer als ein Narren.
Wenn die Fee eine Runde Anspielt und nur Narren geworfen werden hat der erste Narre den Stich geholt und nicht die Fee.

Die Fee wird zur stärksten Karte, wenn sie mit einem Drachen zusammen in einem Stich liegt.
Dabei spielt die Reihenfolge der Karten keine Rolle, also ob die Fee, oder der Drache zuerst gespielt wurde.

\subsection{Bombe}
Wenn die Bombe in einem Stich ist, bekommt \textbf{niemand} diesen.
Der Stich wird aus dem Spiel entfernt und niemand kann dafür Punkte bekommen, oder verlieren.
Die Person, die den Stich bekommen hätte, wenn die Bombe als Narre gewertet werden würde muss die nächste Runde anspielen. % hier kann es vorkommen, dass die Bombe eine Runde anspielen muss

Liegt die Bombe als Trumpffarbe aus, müssen alle ihre Vorhersage machen, \textbf{ohne} ihre Karten anzuschauen.
Dafür ist es generell wichtig, dass niemand seine/ihre Karten anschaut, bevor nicht die Trumpffarbe ausgelegt wurde.

\subsection{Werwolf}
Der Werwolf darf zu jeder Zeit gespielt werden, auch wenn die Runde noch nicht angefangen hat, oder auch wenn sie gerade beendet wurde, oder auch mitten in einer Runde.
Es ist nur wichtig, dass klar angesagt wird, dass diese Karte gerade gespielt wird, sie darf also nicht rein gemogelt werden.
Am besten einigt ihr euch auf ein Codeword, wie z.B. "Stopp".

Der-/Die Spiler:in, der/die den Werwolf ausgespielt hat, muss die Trumpfkarte aus der Mitte nehmen und darf eine neue Trumpffarbe bestimmen (die Aktuelle Trumpffarbe und Weiß sind auch erlaubt).
Danach muss der/die Werfolf-Spiler:in eine Karte aus ihrer Hand spielen.
Er/Sie darf auch die gerade aufgenommene Trumpfkarte spielen.

Wenn der Werwolf als Trumpf ausliegt, gibt es keine Trumpffarbe.

Wenn man den Werwolf spielt, um einen Stich weg zu schanppen, nachdem der Stich bereits pyhsisch aufgenommen wurde, darf dieser nicht mehr weggeschnappt werden.
Der/Die Werfolf-Spilerin hat in dieser Situation zu langsam reagiert.

\subsection{Jongluer}
Sobald der Jongluer gespielt wurde, muss anschließend nach \textbf{jeder} Runde eine Karte an den linken Nachbarn weiter gegeben werden.
Da jede:r Spiler:in dies tut, hat nach jedem Tausche jede:r Spiler:in gleich viele Karten auf der Hand.

Liegt der Jongluer als Trumpfkarte aus, muss von Anfang an jongliert werden.
Dies wird unterbrochen, sobald der Jongluer von dem Werwolf aufgenommen wird.
Danach wird das jonglieren eingestellt, bis dieser wieder gespielt wird.

Ist der Jongluer mit der Bombe zusammen in einem Stich, wird weiter jongliert, da der Jongluer keinen Spilerbezogenen Effekt hat/ nicht beim Aufnehmen aktiviert wird (wie die Wolke), sondern direkt beim Spielen aktiviert wird.

\subsection{Wolke}
Die Person, die den Stich mit der Wolke macht, \textbf{muss} ihre Vorhersage um $+1$, oder $-1$ korrigieren.

Wenn die Wolke mit der Bombe zusammen kommt, hat sie keinen Effekt, da dieser nur die/den Spieler:in betrifft, der/die den Stich macht.
Da mit der Bombe niemand den Stich bekommt, trifft dieser Effekt auch niemanden.


\section{Super Sonderregeln}
\ldots

\end{document}
